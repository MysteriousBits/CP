\chapter{Templates and Scripts}

\section{template.cpp}
\begin{lstlisting}[language=C++]
#include <bits/stdc++.h>
using namespace std;

// Template
// ==============================================
    // pbds
    // #include <ext/pb_ds/assoc_container.hpp>
    // #include <ext/pb_ds/tree_policy.hpp>
    // using namespace __gnu_pbds;
    // template<typename T, typename comp = less<T>>
    // using ordered_set =  tree<T, null_type, comp, rb_tree_tag, tree_order_statistics_node_update>;

    // Debugging
    #ifdef LOCAL
    #include "debug.h"
    #else
    #define debug(...)
    #define see(x)
    #endif

    typedef long long ll;
    typedef vector<int> VI;
    typedef vector<long long> VLL;
    typedef vector<bool> VB;
    typedef vector<vector<int>> VVI;
    typedef pair<int, int> PI;
    typedef pair<ll, ll> PLL;
    typedef vector<pair<int, int>> VPI;

    #define pb push_back
    #define ff first
    #define ss second
    #define mp make_pair
    #define all(a) a.begin(), a.end()
    #define revall(a) a.rbegin(), a.rend()

    #define loop(i, s, e) for (int i = s; i < e; ++i)
    #define inp(v) for (auto& x : v) cin >> x
    #define outp(v) for (int i = 0, n = v.size(); i < n; ++i) cout << v[i] << " \n"[i == n - 1]

    #define nl "\n"
    #define yep cout << "YES\n"
    #define nope cout << "NO\n"

    #define INF (int) 1e9
    #define INFL (ll) 1e18
    // #define MOD 998244353
    #define MOD 1000000007
    #define MAXN 300002
// ==============================================

void solve()
{   
}

int main()
{
    ios_base::sync_with_stdio(false);
    cin.tie(NULL);

    int t = 1;
    cin >> t;
    while(t--) solve();

    #ifdef LOCAL
    cerr << "Execution time: " << 1000.f * clock() / CLOCKS_PER_SEC << " ms." << nl;
    #endif
    
    return 0;
}
\end{lstlisting}
\sectionend

\section{debug.h}
\begin{lstlisting}[language=C++]
#include <iostream>
using namespace std;

#define see(x) cerr << #x << ": " << x << nl

template<typename... Args>
void debug(Args... args)
{
    ((cerr << " " << args), ...) << "\n";
}
\end{lstlisting}
\sectionend

\section{sublime.build}
\begin{lstlisting}
{
    "shell_cmd": "g++ -Wall -std=c++20 -DLOCAL $file -o $file_base_name && ./$file_base_name <input.in> output.out 2> error.log",
    "file_regex": "^(..[^:]*):([0-9]+):?([0-9]+)?:? (.*)$",
    "working_dir": "${file_path}",
    "selector": "source.c++, source.c, source.cpp"
}
\end{lstlisting}
\sectionend

\section{tasks.json (vs code)}
\begin{lstlisting}
{
    "version": "2.0.0",
    "tasks": [
        {
            "label": "cp",
            "type": "shell",
            "command": "",
            "args": [
                "g++",
                "-Wall",
                "-std=c++17",
                "-DLOCAL",
                "-o",
                "${fileBasenameNoExtension}",
                "${fileBasenameNoExtension}.cpp",
                "&&",
                "${fileBasenameNoExtension}",
                "<input.in>",
                "output.out",
                "2>error.log"
            ],
            "group": "build",
            "presentation": {
                "reveal": "silent"
            },
            "problemMatcher": {
                "owner": "cpp",
                "fileLocation": ["relative", "${workspaceRoot}"],
                "pattern": {
                "regexp": "^(.*):(\\d+):(\\d+):\\s+(warning|error):\\s+(.*)$",
                "file": 1,
                "line": 2,
                "column": 3,
                "severity": 4,
                "message": 5
                }
            }
        }
    ]
}
\end{lstlisting}
\sectionend

