\chapter{Data Structure}

\section{Fenwick Tree}
\begin{lstlisting}[language=C++]
struct fenwick
{
    int n;
    vector<ll> tree;

    fenwick(int s)
    {
        n = s;
        tree.assign(n + 1, 0);
    }

    ll get(int i)
    {
        ll res = 0;
        for(; i > 0; i -= i & -i) res += tree[i];
        return res;
    }

    void add(int i, int x)
    {
        for (; i <= n; i += i & -i) tree[i] += x;
    }
    
    ll rsq(int l, int r)
    {
        return get(r) - get(l - 1);
    }
};
\end{lstlisting}
\sectionend

\section{Segment Tree}
\begin{lstlisting}[language=C++]
template<typename T>
struct segtree
{
    #define left(u) 2 * u + 1
    #define right(u) 2 * u + 2

    int n;
    T init_val;
    vector<T> tree;
    function<T (T, T)> f;

    bool is_lazy = false;
    T init_lazy;
    vector<T> lazy;
    function<void (T&, T, int, int)> apply_lazy;
    function<void (T&, T)> merge_lazy;

    segtree() {} // Dummy constructor for global declaration
    
    segtree(int sz, function<T (T, T)> f, T val = 0)
    {
        n = sz;
        this->f = f;
        init_val = val;

        tree.assign(1 << (__lg(n - 1) + 2), init_val);
    }

    // apply(tree[u], lazy[u], l, r)
    // merge(lazy[left(u)], lazy[u])
    void make_lazy(function<void (T&, T, int, int)> apply, function<void (T&, T)> merge, T init = 0)
    {
        is_lazy = true;
        init_lazy = init;
        apply_lazy = apply;
        merge_lazy = merge;

        lazy.assign(tree.size(), init_lazy);
    }

    void build(vector<T>& a) { _build(0, 0, n - 1, a); }
    void update(int i, T x) { _update(0, 0, n - 1, i, x); }
    void update(int l, int r, T x) { _update_range(0, 0, n - 1, l, r, x); }
    T get(int l, int r) { return _get(0, 0, n - 1, l, r); }

    void _build(int u, int l, int r, vector<T>& a)
    {
        if (l == r)
        {
            tree[u] = a[l];
            return;
        }

        int mid = (l + r) / 2;
        _build(left(u), l, mid, a);
        _build(right(u), mid + 1, r, a);
        tree[u] = f(tree[left(u)], tree[right(u)]);
    }

    void _update(int u, int l, int r, int i, T x)
    {
        if (l == r)
        {
            tree[u] = x;
            return;
        }

        int mid = (l + r) / 2;
        if (i <= mid) _update(left(u), l, mid, i, x);
        else _update(right(u), mid + 1, r, i, x);
        tree[u] = f(tree[left(u)], tree[right(u)]);
    }

    void propagate(int u, int l, int r)
    {
        if (lazy[u] == init_lazy || l == r) return;

        apply_lazy(tree[left(u)], lazy[u], l, r);
        merge_lazy(lazy[left(u)], lazy[u]);
        apply_lazy(tree[right(u)], lazy[u], l, r);
        merge_lazy(lazy[right(u)], lazy[u]);

        lazy[u] = init_lazy;
    }

    void _update_range(int u, int tl, int tr, int l, int r, T x)
    {
        if (l > tr || r < tl) return;
        if (tl >= l && tr <= r)
        {
            apply_lazy(tree[u], x, tl, tr);
            merge_lazy(lazy[u], x);
            return;
        }

        propagate(u, tl, tr);

        int mid = (tl + tr) / 2;
        _update_range(left(u), tl, mid, l, r, x);
        _update_range(right(u), mid + 1, tr, l, r, x);

        tree[u] = f(tree[left(u)], tree[right(u)]);
    }

    T _get(int u, int tl, int tr, int l, int r)
    {
        if (l > tr || r < tl) return init_val;

        if (l <= tl && r >= tr) return tree[u];

        if (is_lazy) propagate(u, tl, tr);

        int mid = (tl + tr) / 2;
        return f(_get(left(u), tl, mid, l, r),
                 _get(right(u), mid + 1, tr, l, r));
    }

    #undef left
    #undef right
};
\end{lstlisting}
\sectionend

\section{DSU}
\begin{lstlisting}[language=C++]
struct dsu
{
    vector<int> parent, size;
 
    dsu(int n)
    {
        parent.resize(n + 1);
        size.resize(n + 1);
 
        for (int i = 1; i <= n; ++i)
        {
            parent[i] = i;
            size[i] = 1;
        }
    }

    int find(int u)
    {
        if (parent[u] == u) return u;
        return parent[u] = find(parent[u]);
    }
 
    void merge(int u, int v)
    {
        u = find(u);
        v = find(v);
 
        if (u != v)
        {
            if (size[u] < size[v]) swap(u, v);
            parent[v] = u;
            size[u] += size[v];
        }
    }
};
\end{lstlisting}
\sectionend
